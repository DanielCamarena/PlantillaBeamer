%------------------------------------------------
\section{Conceptos Previos} 

%------------------------------------------------
\begin{frame}
\frametitle{Lógica matematica}

\textbf{Lógica de primer orden}: es un sistema formal diseñado para estudiar la inferencia en los lenguajes de primer orden. 

\textbf{Lenguaje de primer orden}: es un lenguaje formal (español), cuyos símbolos primitivos y reglas para unir esos símbolos están especificados con conectores lógicos, cuantificadores y funciones proposicionales. %\\~

\pause
\begin{itemize}
 \item Axioma: enunciado fundamental
 \item Concepto primitivo: concepto no definido
 \item Demostración: secuencia finita de enunciados que conducen de un conjunto de premisas a una sentencia dada
 \item Teorema: proposición derivada como la conclusión de una demostración
\end{itemize}
\end{frame}

\begin{frame}
\frametitle{Ejemplos}

 En la teoría de conjuntos ZF/ZFC:
\begin{itemize}
 \item Axioma: existe el conjunto vacío 
 \item Concepto primitivo: conjunto 
 \item Teorema: no existe el conjunto que es elemento de sí mismo 
\end{itemize}

\end{frame}

%------------------------------------------------

\begin{frame}
\frametitle{Ejemplo de demostración}

En la aritmética de Peano:
\begin{itemize}
 \item {\color{red} \textbf{Teorema}: Para todo $m$ y $n$ enteros positivos, si $m$ y $n$ son pares, entonces $m+n$ es par.}
 \item \textbf{Demostración}: Supongamos que $m$ y $n$ son enteros pares arbitrariamente elegidos. [Debe mostrarse que $m+n$ es par.]\\
  \begin{enumerate}[1.]
  	\item $m = 2r, n = 2s$ para algunos enteros $r$ y $s$ (por definición de par)
  	\item $m + n = 2r + 2s$ (por sustitución)
  	\item $m + n = 2(r + s)$ (mediante la factorización de $2$)
  	\item $r + s$ es un entero (pues es la suma de dos enteros)
  	\item $m + n$ es par (por definición de par)
  \end{enumerate}
\end{itemize}
\end{frame}

%------------------------------------------------

\begin{frame}
\frametitle{Teoría}

Sistema hipotético deductivo formado por un conjunto de proposiciones dentro de un lenguaje formal.\\

\pause
\begin{itemize}
 \item Consistente: si para cada par de fórmulas $(\varphi,\neg \varphi)$ solo una pertenece a la teoría
 \item Completa: si para cada par de fórmulas $(\varphi,\neg \varphi)$ al menos una pertenece a la teoría
\end{itemize}
\end{frame}

\begin{frame}
	\frametitle{Teoría}
			
\begin{center}
	¡¡En matemáticas todas las teorías son consistentes!!\\
	\dots\\
	Pero no todas son completas
\end{center}

\pause
Ejemplos:
\begin{itemize}
	\item La teoría de conjuntos ZFC 
	\item La teoría de la selección natural (no es teoría lógica, es científica)
\end{itemize}

\end{frame}

\begin{frame}
\frametitle{Paradoja}

Son argumentos donde
\begin{itemize}
 \item hay premisas no controvertidas y verdaderas.
 \item emplea un procedimiento no controversial.
 \item obtiene una conclusión
 \pause
 \begin{itemize}
  \item contradictoria
  \item absurda, inapropiada o inaceptable
 \end{itemize}
\end{itemize}

\vfill

\pause
Ejemplo: \href{https://es.wikipedia.org/wiki/Paradoja_de_Banach-Tarski}{La paradoja de Banach-Tarski}
\end{frame}
